\documentclass{article}
\usepackage[UTF8]{ctex}
\setmainfont{Calibri Light}
\usepackage{setspace}
\renewcommand{\baselinestretch}{1.2}
\usepackage{amsmath}
\usepackage{amssymb}
\usepackage{ntheorem}
\usepackage{graphicx}
\usepackage{bbm}
\usepackage{hyperref}
\hypersetup{
	colorlinks=true,
	linkcolor=blue,
	filecolor=cyan,      
	urlcolor=red,
	citecolor=green,
}
\newtheorem{theorem}{Theorem}
\newtheorem{corollary}{Corollary}
\newtheorem{lemma}{Lemma}
\newtheorem*{proof}{Proof}
\setlength{\parindent}{2em}
\author{Siheng Zhang\\zhangsiheng@cvte.com}
\title{第\textbf{\textit{1}}章\ \ \ \ 概率近似正确}
\date{\today}      
\usepackage[a4paper,left=18mm,right=18mm,top=25mm,bottom=25mm]{geometry} 

\begin{document}
\maketitle  


本章对应 \textbf{UML第2-5章},主要回答如下问题:

\begin{itemize}
\item 关于泛化误差,我们能知道什么?
\item 当我们谈论归纳偏置的时候,我们在谈论什么?
\end{itemize}

\tableofcontents
\newpage

\section{学习器形式化:输入、输出与评价}

\begin{itemize}

	\item \textbf{输入}:即训练集,为一个有限的序列$S=((x_1,y_1),\cdots,(x_m,y_m))$。训练集为实例集和标签集的笛卡尔积,其中实例$x \in \mathcal{X}$,标签 $y \in \mathcal{Y}$。本章主要考虑二分类问题,即$\mathcal{Y}=\{0,1\}$。
	
	\item \textbf{输出}:假设(hypothesis),有时候称为分类器(classifier)、回归器(regressor) $h:\mathcal{X}\rightarrow\mathcal{Y}$。

	\item \textbf{数据生成模型}:假设实例依概率分布$\mathcal{D}$产生,并且有一个“绝对正确”的标签生成函数(至少目前我们假设如此)$f:\mathcal{X}\rightarrow\mathcal{Y}$。
	
	独立同分布(i.i.d)假设:训练样本的产生独立地来自于同一个分布。
	
	\item \textbf{泛化误差}:也称为真实误差或者真实风险。以$\mathcal{D}(A)$表示样例$x \in A$的概率,则泛化误差为:
	
	\begin{equation}
	L_{\mathcal{D},f}(h)\overset{def}{=}\mathop{\mathbb{P}}\limits_{x\sim\mathcal{D}}[h(x)\neq f(x)]\overset{def}{=}\mathcal{D}(\{x:h(x)\neq f(x)\})
	\end{equation}
	
\end{itemize}

	\vspace{1mm}
	\begin{scriptsize}
	\begin{spacing}{1.2}
	{\sffamily
	\noindent\textit{\underline{注1.} 数据生成模型对学习器不可见。}

    \noindent\textit{\underline{注2.} 通常我们说“训练集”,但是严格来说应该说“训练序列”,因为同样的样本可以多次出现,并且一些学习算法的输出因输入顺序不同而不同。}
    
    \noindent\textit{\underline{注3.} 严格来说,$\mathcal{D}$定义在$\mathcal{X}\times\mathcal{Y}$上,但通常不严格区分。}}
	\end{spacing}
	\end{scriptsize}
	\vspace{-3mm}
	
\section{从最小经验风险到概率近似正确}

\subsection{最小经验风险(Empirical Risk Minimization,ERM) 准则可能造成过拟合}

	显然,泛化误差是不可知的。因此,我们转而最小化经验风险:

	\begin{equation}
	L_S(h)\overset{def}{=}\frac{|\left\{(x_i,y_i)\in S:h(x_i)\neq y_i \right\}|}{m}
	\end{equation}

	考虑一个“懒惰”的学习器$h$,它记住了所有的样本,当且仅当$x=x_i$时候输出$y=y_i$,其它时候都输出0。显然,它的经验风险$L_S(h)=0$,但是对于未见样本,它有一半的概率预测失败,$L_{\mathcal{D},f}(h)=1/2$。这种在训练集上表现特别好,但是泛化性能很差的现象,称为“过拟合”。这个例子蕴含着一个很重要的基本事实:
	
	\textbf{如果不对假设集加以限制,最小经验风险准则可能造成过拟合。}

\subsection{归纳偏置(inductive bias)下的最小经验风险准则}

	选取假设集即反映了人对任务的先验知识,即归纳偏置。记假设集为$\mathcal{H}$,最小经验风险准则表述为:
	\begin{equation}
	h_S\in \arg\min \limits_{h\in\mathcal{H}}L_S(h)
	\end{equation}

	“理想”的情况下,假设集中包含有泛化误差为0的假设,即存在$h^*\in\mathcal{H}$使得$L_{\mathcal{D},f}(h^*) = 0$,我们称之为\textbf{可实现假设}。该条件蕴含着$L_S(h^*)=0$, $L_S(h_S)=0$,而事实上,我们更感兴趣的是$L_{\mathcal{D},f}(h_S)$。

\subsection{概率近似正确(Probably Approximately Correct,PAC)可学习性}
\noindent\textbf{定义}:称假设集是概率近似正确可学习的,当给定大小为$m\geq m_\mathcal{H}(\epsilon,\delta)$的样本集,假设集中存在一个假设以至少$1-\delta$的置信度达到不低于$1-\epsilon$的正确率,即$P(L_{\mathcal{D},f}(h_S) < \epsilon) > 1-\delta$。
	
	\begin{theorem} 有限假设集是PAC可学习的,且样本复杂度为$m_\mathcal{H}(\epsilon,\delta)=\frac{\log(|\mathcal{H}|/\delta)}{\epsilon}$.
	\end{theorem}
	
	\begin{proof} 在假设集中存在着一些糟糕的假设,它们的泛化误差高于一定阈值。形式化来说,记糟糕的假设集为$\mathcal{H}_B$,它是$\mathcal{H}$的子集,并且$\forall h\in\mathcal{H}_B,L_{\mathcal{D},f}(h)>\epsilon$。而对于某些数据集,这样的假设仍然能有良好的训练性能,记为$M=\{S:\exists h\in \mathcal{H}_B, L_S(h)=0\}$。注意到,$M=\bigcup\limits_{h\in\mathcal{H}_B}\{S:L_S(h)=0\}$。而我们的证明目标是事件$L_{\mathcal{D},f}(h_S)>\epsilon$的概率上界,即
	\begin{equation*}
	\begin{split}
	&\mathcal{D}^m(\{S:L_{\mathcal{D},f}(h_S)>\epsilon\})\leq\mathcal{D}^m(M) 
	=\mathcal{D}^m(\bigcup\limits_{h\in\mathcal{H}_B}\{S:L_S(h)=0\})        \\ 
	&=\sum_{h\in\mathcal{H}_B}\prod_{i=1}^m\mathcal{D}(\{x_i:f(x_i)=h(x_i)\})
	\overset{i.i.d.}{=}\sum_{h\in\mathcal{H}_B}(1-L_{\mathcal{D},f}(h))^m      \\
	&\leq\sum_{h\in\mathcal{H}_B}(1-\epsilon)^m\leq\sum_{h\in\mathcal{H}_B}\exp(-\epsilon m)
	\leq|\mathcal{H}|\exp(-\epsilon m)
	\end{split}
	\end{equation*}

	令$|\mathcal{H}|\exp(-\epsilon m)\leq\delta$,可以得到$m\geq\log(|\mathcal{H}|/\delta)/\epsilon$。
	\end{proof}

\subsection{“没有免费午餐”(No-Free-Lunch,NFL)定理}

	\begin{theorem}
	令$A$表示0-1损失下二分类任务的任一学习算法,假设训练集规模为$m<|\mathcal{X}|/2$。则存在一个定义在${X}\times\{0,1\}$上的分布$\mathcal{D}$使得:随机采样训练集$S\sim\mathcal{D}^m$,$P(L_\mathcal{D}(A(S))\geq 1/8)\geq 1/7$。
	\end{theorem}
	
	\begin{proof}
	令$C\subseteq\mathcal{X}$含有$2m$个样本,则标签函数存在$2^{2m}$种可能,记为$f_1,\cdots,f_T$。对任一标签函数,定义在$\mathcal{C}\times\{0,1\}$上的分布$\mathcal{D}_i$:
	
	\begin{equation*}
	\mathcal{D}_i(\{(x,y)\})=\left\{
	\begin{matrix} 1/|C|,& \mathrm{if}\ y=f_i(x) \\ 0,& \mathrm{otherwise} 
	\end{matrix}\right.
	\end{equation*}
	
	从$C$中获取$m$个样本构成数据集,共有$k=(2m)^m$个可能的数据集。记由函数$f_i$打标签的数据集为$\mathcal{S}_j^i, j=1,2,\cdots,k$,因此
	\begin{equation*}
	\mathop{\mathbb{E}}\limits_{S\sim\mathcal{D}_i^m}[L_{\mathcal{D}_i}(A(S))]=\frac{1}{k}\sum_{j=1}^kL_{\mathcal{D}_i}(A(S_j^i))
	\end{equation*}
	
注意到	
	\begin{equation*}
	\max_{i\in\{1,\cdots,T\}} \frac{1}{k}\sum_{j=1}^k L_{\mathcal{D}_i}(A(S_j^i)) \geq \frac{1}{T}\sum_{i=1}^T \frac{1}{k} \sum_{j=1}^k L_{\mathcal{D}_i}(A(S_j^i)) \geq \min_{j\in\{1,\cdots,k\}}\frac{1}{T}\sum_{i=1}^T L_{\mathcal{D}_i} (A(S_j^i))
	\end{equation*}

	记$S_j=(x_1,\cdots,x_m)$,同时记$v_1,\cdots,v_p$为$C$中未出现在$S_j$的样本。显然,$p\geq m$(当$S_j$中存在重复样本的时候,不等号严格成立)。因此,对任意$h:C\rightarrow\{0,1\}$,
	
	\begin{equation*}
	L_{\mathcal{D}_i}(h)=\frac{1}{2m}\sum_{x\in C}\mathbbm{1}_{[h(x)\neq f_i(x)]}\geq\frac{1}{2p}\sum_{r=1}^p\mathbbm{1}_{[h(v_r)\neq f_i(v_r)]}
	\end{equation*}

	\begin{equation*}
	\frac{1}{T}\sum_{i=1}^T L_{\mathcal{D}_i} (A(S_j^i))\geq\frac{1}{2}\min_{r\in\{1,\cdots,p\}}\frac{1}{T}\mathbbm{1}_{[A(S^i_j)(v_r)\neq f_i(v_r)]}
	\end{equation*}
其中$r\in[p]$。

	将标签函数$f_1,\cdots,f_T$切分成$T/2$对不相交的函数对,对于其中的某对函数$(f_i,f_{i'})$,有$\forall c\in C,f_i(c)\neq f_{i'}(c)$当且仅当$c=v_r$。则它们对数据集$S_j$给出了同样的标签,因而$S_j^i=S_j^{i'}$。从而有, $\mathbbm{1}_{[A(S_j^i)(v_r)\neq f_i(v_r)]}+\mathbbm{1}_{[A(S_j^{i'})(v_r)\neq f_{i'}(v_r)]}=1$。这意味着
	
	\begin{equation*}
	\frac{1}{T}\sum_{i=1}^T\mathbbm{1}_{[A(S_j^i)(v_r)\neq f_i(v_r)]}=\frac{1}{2}
	\end{equation*}

	至此,

	\begin{equation*}
	\max_{i\in\{1,\cdots,T\}}\mathop{\mathbb{E}}_{S\sim\mathcal{D}_i^m}[L_{\mathcal{D}_i}(A(S))]\geq 1/4
	\end{equation*}

	这说明了,存在某些标签函数和分布$f,\mathcal{D}$使得$L_\mathcal{D}(f)=0$,且$\mathop{\mathbb{E}}\limits_{S\sim\mathcal{D}^m}[L_\mathcal{D}(A(S))]\geq 1/4$。而后者可以进一步推出违背PAC可学习性的结论。注意到,对一个随机变量$\theta\in[0,1]$,如果满足$\mathbb{E}(\theta)\geq 1/4$,则有:
	
	\begin{equation*}
	p\left(\theta\geq\frac{1}{8}\right)=\int_\frac{1}{8}^1 p(\theta) \mathrm{d}\theta \geq\int_\frac{1}{8}^1 \theta p(\theta) \mathrm{d}\theta=\mathbb{E}(\theta)-\int_0^\frac{1}{8}\theta p(\theta)\mathrm{d}\theta \geq\frac{1}{4}-\frac{1}{8}\int_0^\frac{1}{8} p(\theta)\mathrm{d}\theta \Rightarrow p \left(\theta\geq\frac{1}{8}\right)\geq \frac{1}{7}
	\end{equation*}
	\end{proof}

	\textbf{NFL定理从数学形式上确认了归纳偏置的必要性。}

\section{不可知情况下的PAC(Agnostic PAC, A-PAC)可学习性}

	在实际情况中,也许不存在所谓正确的标签函数,标签也不一定可以由手头的特征决定。因此可实现假设是不成立的,称之为不可知情况。此时,PAC可学习性需要重新定义。称假设集是A-PAC可学习的,当在规模为$m\geq m_\mathcal{H}(\epsilon,\delta)$的训练集上训练时,存在一个算法,以至少$1-\delta$的置信度成立如下条件:
	
	\begin{equation*}
	L_\mathcal{D}(h)\leq\min\limits_{h'\in\mathcal{H}}L_\mathcal{D}(h')+\epsilon
	\end{equation*}
其中$L_\mathcal{D}(h)\overset{def}{=}\mathop{\mathbb{P}}\limits_{(x,y)\sim\mathcal{D}}[h(x)\neq y]\overset{def}{=}\mathcal{D}(\{x:h(x)\neq y\})$.

	此外,损失函数也可以拓展到一般情况,仅需将二分类下单个样本的损失函数拓展到一般度量,$L_\mathcal{D}(h)=\mathop{\mathbb{E}}\limits_{x\sim\mathcal{D}}[l((x),y)]$。
	
\noindent\textbf{定义} ($\epsilon$-典型性):在定义域$Z$,假设集$\mathcal{H}$,损失函数$l$和分布$\mathcal{D}$给定的前提下,称训练集$S$具有$\epsilon$-典型性,如果其满足,$\forall h\in\mathcal{H},|L_S(h)-L_\mathcal{D}(h)|\leq\epsilon$。
	
	\begin{lemma}
	假设训练集$S$具有$\epsilon/2$-典型性,则任一$h_S\in\arg\min_{h\in\mathcal{H}}L_S(h)$都满足$L_\mathcal{D}(h_S) \leq L_\mathcal{D}(h)+\epsilon$。
	\end{lemma}
	
	\begin{proof}
	对任一$h\in\mathcal{H}$,$L_\mathcal{D}(h_S) \leq L_S(h_S)+\frac{\epsilon}{2} \leq L_S(h)+\frac{\epsilon}{2} \leq L_\mathcal{D}(h)+\frac{\epsilon}{2}+\frac{\epsilon}{2}\leq L_\mathcal{D}(h)+\epsilon$。
	\end{proof}
	
	该引理说明了,ERM规则在$\epsilon/2$-典型的数据集上能够成功返回好的假设。于是,要使得ERM规则是否适用于A-PAC,就只要保证随机选取一个训练集,其为$\epsilon$-典型的概率至少为$1-\delta$。可想而知,决定随机选取的数据集是否足够典型的一个先决条件是其大小。下面的定义对其所需的大小进行了刻画。
	
\noindent\textbf{定义} (一致收敛性):称假设集$\mathcal{H}$具有一致收敛性,如果存在函数$m^{UC}_\mathcal{H}:(0,1)^2\rightarrow \mathbb{N}$,使得对任意$\epsilon,\delta \in (0, 1)$和定义在$Z$上的分布$\mathcal{D}$,数据集$S$的样本独立同分布地采样自$\mathcal{D}$,$S$的大小满足$m\geq m^{UC}_\mathcal{H}(\epsilon, \delta)$,则$S$为$\epsilon$-典型的概率至少为$1-\delta$。

	\begin{corollary}
	如果一个假设集$\mathcal{H}$在函数$m^{UC}_\mathcal{H}$下具有一致收敛性,则其在训练集大小满足$m_\mathcal{H}(\epsilon,\delta)\leq m^{UC}_\mathcal{H}(\epsilon/2,\delta)$条件下A-PAC可学习,且ERM$_\mathcal{H}$策略是$\mathcal{H}$成功的A-PAC学习器.
	\end{corollary}
	
	\begin{lemma}
	(Hoeffding不等式) 令$\theta_1,\cdots,\theta_m$为一序列独立同分布的随机变量,假设$\forall i, \mathbb{E}[\theta_i]=\mu$ 且$P[a\leq\theta_i\leq b]=1$,则对任一$\epsilon>0$,
	\begin{equation*}
	P\left[\left| \frac{1}{m}\sum_{i=1}^m\theta_i - \mu \right| >\epsilon \right] \leq 2 \exp \left( \frac{-2m\epsilon^2 }{(b-a)^2} \right)
	\end{equation*}
	\end{lemma}

	\begin{theorem}
	\textbf{A-PAC样本复杂度} 假设损失函数的取值范围为$[a,b]$,则满足有限假设集$\mathcal{H}$是A-PAC可学习的样本复杂度为
	
	\begin{equation}
	m_\mathcal{H}(\epsilon,\delta)\leq m^{UC}_\mathcal{H}(\epsilon/2,\delta)\leq\left\lceil\frac{2\log(2|\mathcal{H}|/\delta)(b-a)^2}{\epsilon^2}\right\rceil
	\end{equation}
	
	\begin{proof}
	固定$\epsilon,\delta$,目的是找到样本数量$m$来保证:对任一分布$\mathcal{D}$,样本集$S\overset{i.i.d.}{\sim}\mathcal{D}^m$,使得
	\begin{equation*}
	\mathcal{D}^m(\{S:\exists h\in\mathcal{H}, |L_S(h)-L_\mathcal{D}(h)|>\epsilon\}) < \delta
	\end{equation*}
式子左侧可以放缩为:
	\begin{equation*}
	\mathcal{D}^m(\{S:\exists h\in\mathcal{H}, |L_S(h)-L_\mathcal{D}(h)|>\epsilon\}) \leq \sum_{h\in\mathcal{H}} \mathcal{D}^m(\{S:|L_S(h)-L_\mathcal{D}(h)|>\epsilon\})
	\end{equation*}

	由于样本独立地采样自$\mathcal{D}$,由期望地线性性,$\mathbb{E}(L_S(h))=\frac{1}{m}\sum_{i=1}^m \mathbb{E}(l(h(x_i),y_i))=L_\mathcal{D}(h)$。因此,$|L_\mathcal{D}(h)-L_S(h)|$可以视为随机变量$L_S(h)$和其期望之间的距离。应用Hoeffding不等式可以得到,
	\begin{equation*}
	\mathcal{D}^m(\{S:|L_S(h)-L_\mathcal{D}(h)|>\epsilon\}) \leq 2 \exp \left( \frac{-2m\epsilon^2 }{(b-a)^2} \right)
	\end{equation*}
	
	结合上述方程可以得到
	\begin{equation*}
	\mathcal{D}^m(\{S:\exists h\in\mathcal{H}, |L_S(h)-L_\mathcal{D}(h)|>\epsilon\}) \leq \sum_{h\in\mathcal{H}} 2 \exp \left( \frac{-2m\epsilon^2 }{(b-a)^2} \right) = 2|\mathcal{H}|\exp \left( \frac{-2m\epsilon^2 }{(b-a)^2} \right)
	\end{equation*}
	选取$m\geq\frac{2\log(2|\mathcal{H}|/\delta)(b-a)^2}{\epsilon^2}$,则方程左侧的上界为$\delta$。
	\end{proof}	
	\end{theorem}

\section{误差分解}

	\begin{equation}
	L_\mathcal{D}(h_S) = \underbrace{\min\limits_{h\in\mathcal{H}}L_\mathcal{D}(h)}_{\epsilon_{\mathrm{app}}}
									+ \underbrace{L_\mathcal{D}(h_S)-\epsilon_{\mathrm{app}}}_{\epsilon_{\mathrm{est}}}
	\end{equation}		

	\begin{itemize}
	\item \textbf{逼近误差}衡量了选取当前假设集(即归纳偏置)的风险,注意其与数据集规模无关。扩大假设集的规模可以减少逼近误差。
	\item \textbf{拟合误差}衡量了经验风险(即训练误差),是真实风险的一个经验估计。拟合误差取决于训练集的大小(随其增大而减小)与假设集的规模(随其增大而对数上升)。
	\end{itemize}

\section{总结}

	\vspace{1mm}
	\begin{scriptsize}
	\begin{spacing}{1.2}
	{\sffamily
	\noindent\textit{\underline{注.} 在本系列笔记的整理过程中,这一小节的出现意味着,这章枯燥且乏味,但这章的偏理论的内容是不需要记住的,因为这些推导所指向的下述结论是非常浅显的。小结最后还会写到,这章的内容将如何与其它章节前后关联,以使读者明白其在整个机器学习理论中所处的地位。如果某些章没有这一小节,那么说明该章节的内容浅显易懂,不需要这一小节。)
	}}
	\end{spacing}
	\end{scriptsize}
	\vspace{1mm}

	至此,在PAC学习理论框架下,我们有了如下重要结论:
\begin{enumerate}
\item  不存在对一切问题通用的学习器;
\item  归纳偏置对于防止过拟合是必要的;
\item  样本复杂度函数跟假设集规模、置信度和误差水平有关,但有趣的是,它和样本空间并无关联;
\item  归纳偏置控制着拟合误差和逼近误差的折衷。
\end{enumerate}

	现在,我们触及了学习理论的核心问题:\textbf{在什么样的假设集下,ERM规则不会过拟合(即PAC可学习)?} 当前,我们仅保证了在有限假设集上的性质。下一章谈及VC维时候,这个问题才会有一个更加精确的答案。

\section{练习与答案}

\begin{itemize}
\item[Ex1] (UML Ex2.2) Let $\mathcal{H}$ be a class of binary classifiers over a domain $\mathcal{X}$. Let $\mathcal{D}$ be an unknown distribution over $\mathcal{X}$, and let $f$ be the target hypothesis in $\mathcal{H}$. Fix some $h\in\mathcal{H}$, show that the expected value of $L_S(h)$ over the choice of $S$ equals $L_{\mathcal{D},f}(h)$, namely,

\begin{equation*}
\mathop\mathbb{E}\limits_{S\sim\mathcal{D}^m}[L_S(h)]=L_{\mathcal{D},f}(h)
\end{equation*}

\item[] \textbf{Solution}: By the linearity of expectation,
\begin{equation*}
\begin{split}
\mathop\mathbb{E}\limits_{S\sim\mathcal{D}^m}[L_S(h)] 
&= \mathop\mathbb{E}\limits_{S\sim\mathcal{D}^m}[\frac{1}{m}\sum_{i=1}^m \mathbbm{1}_{[h(x_i)\neq f(x_i)]}] \\
&= \frac{1}{m}\sum_{i=1}^m \mathop\mathbb{E}\limits_{x_i \sim\mathcal{D}^m}[\mathbbm{1}_{[h(x_i)\neq f(x_i)]}] \\
&= \frac{1}{m} \cdot m \cdot L_{\mathcal{D},f}(h) = L_{\mathcal{D},f}(h)
\end{split}
\end{equation*}

\item[Ex2] (UML Ex2.3) \textbf{Axis Aligned rectangles}: An axis aligned rectangle classifier in the plane is a classifier that assigns the value 1 to a point if and only if it is inside a certain rectangle. Formally, given real numbers $a_1\leq b_1,a_2\leq b_2$, define the classifier $h(a_1,b_1,a_2,b_2)$ by:

	\begin{equation*}
	h(a_1,b_1,a_2,b_2)(x_1, x_2) = \left\{\begin{matrix}
	1,& \mathrm{if}\ a_1 \leq x_1 \leq b1\ \mathrm{and}\  a_2 \leq x_2 \leq b_2 \\
	0,& \mathrm{otherwise}
	\end{matrix}\right.
	\end{equation*}

	The class of all axis aligned rectangles in the plane is defined as:
	
	\begin{equation*}
	\mathcal{H}^2_{\mathrm{rec}}=\{h_{(a_1,b_1,a_2,b_2)}:a_1\leq b_1,\ \mathrm{and}\ a_2\leq b_2\}
	\end{equation*}
Note that this is an infinite size hypothesis class. Throughout this exercise we rely on the realizability assumption.

	\begin{itemize}
	\item[2.1] Let $A$ be the algorithm that returns the smallest rectangle enclosing all positive examples in the training set. Show that $A$ is an ERM.
	\item[2.2] Show that if $A$ receives a training set of size $\geq 4\frac{\log(4/\delta)}{\epsilon}$, then, with probability of at least $1-\delta$ it returns a hypothesis with error of at most $\epsilon$.
	
	\textit{Hint: Fix some distribution $\mathcal{D}$ over $\mathcal{X}$, let $R^*=R(a^*_1,b^*_1,a^*_2,b^*_2)$ be the rectangle that generates the labels, and let $f$ be the corresponding hypothesis. Let $a_1\geq a^*_1$ be a number such that the probability mass (with respect to $\mathcal{D}$) of the rectangle $R_1=R(a^*_1,a_1,a^*_2,b^*_2)$ is exactly $\epsilon/4$. Similarly, let $b_1,a_2,b_2$ be numbers such that the probability masses of the rectangles $R_2=R(b_1,b^*_1,a^*_2,b^*_2)$, $R_3=R(a^*_1,b^*_1,a^*_2,a_2)$, $R_4=R(a^*_1,b^*_1,b_2,b^*_2)$ are all $\epsilon/4$. Let $R(S)$ be the rectangle returned by $A$. See illustration in the following figure.}
	
	\begin{figure}[!htbp]
	\center
	\includegraphics[scale=.8]{1.png}
	\end{figure}
	
	\begin{itemize}
	\item Show that $R(S)\subseteq R^*$.
	\item Show that if $S$ contains (positive) examples in all of the rectangles $R_1,R_2,R_3,R_4$, then the hypothesis returned by $A$ has error of at most $\epsilon$.
	\item For each $i\in\{1,\cdots,4\}$, upper bound the probability that $S$ does not contain an example from $R_i$.
	\item Use the union bound to conclude the argument.
	\end{itemize}
	 
	\item[2.3] Repeat the previous question for the class of axis aligned rectangles in $\mathbb{R}^d$.
		
	\item[2.4] Show that the runtime of applying the algorithm $A$ mentioned earlier is polynomial in $d, 1/\epsilon$, and in $\log(1/\delta)$.
	\end{itemize}

	\item[] \textbf{Solution}
	
	\begin{itemize}
	\item[2.1] In realizable setup, since the tightest rectangle enclosing all positive example is returned, all positive and negative instances are correctly classified.
	\item[2.2] By definition, algorithm A returns the tightest rectangle, so $R(S)\subseteq R^*$. 
	\end{itemize}

\item[Ex3] (UML Ex3.2) Let $\mathcal{X}$ be a discrete domain, and let $\mathcal{H}_{\text{Singleton}}=\{h_z:z\in\mathcal{X}\}\cup\{h^-\}$, where for each $z\in\mathcal{X}$, $h_z$ is the function defined by $h_z(x)=1$ if $x=z$ and $h_z(x)=0$ if $x\neq z$. $h^-$ is simply the all-negative hypothesis, namely, $\forall x\in\mathcal{X},h^-(x)=0$. The realizability assumption here implies that the true hypothesis $f$ labels negatively all examples in the domain, perhaps except one.

	\begin{itemize}
	\item[3.1] Describe an algorithm that implements the ERM rule for learning $\mathcal{H}_{\text{Singleton}}$ in the realizable setup.
	\item[3.2] Show that $\mathcal{H}_{\text{Singleton}}$ is PAC learnable. Provide an upper bound on the sample complexity.
	\end{itemize} 

	\item[] \textbf{Solution}:
	\begin{itemize}
	\item[3.1] Traverse $z\in\mathcal{X}$ then output $h_z$ or $h^-$.
	\item[3.2] If for any $i\in[1,\cdots,m]$, $h_{x_i}$ is the true hypothesis, the algorithm can find it in the realizable setup. Otherwise, the algorithm outputs $h^-$, which can be either true or false (i.e., the target $z^*$ is not in the training set). Note that in the second case, the algorithm only makes a single error when generalize to all cases, and hence $p(z^*)\geq\epsilon$ (otherwise, it is meaningless),
	
	\begin{equation*}
	\mathbb{P}(L_{\mathcal{D},f}(h_S)>\epsilon)\leq(1-p(z^*))^m\leq(1-\epsilon)^m\leq\exp(-\epsilon m)\leq\delta
	\end{equation*}

which leads to
	\begin{equation*}
	m_\mathcal{H}(\epsilon,\delta)\leq\left\lceil\frac{\log(1/\delta)}{\epsilon}\right\rceil
	\end{equation*}
	

\item[Ex4] (UML Ex3.3) Let $\mathcal{X}=\mathbb{R}^2$, $\mathcal{Y}=\{0,1\}$, and let $\mathcal{H}$ be the class of concentric circles in the plane, that is, $\mathcal{H}=\{h_r:r\in\mathbb{R}_+\}$, where $h_r(x)=\mathbbm{1}_{[\| x\|\leq r]}$. Prove that $\mathcal{H}$ is PAC learnable (assume realizability), and its sample complexity is bounded by:

	\begin{equation*}
	m_\mathcal{H}(\epsilon,\delta)\leq\left\lceil\frac{\log(1/\delta)}{\epsilon}\right\rceil
	\end{equation*}
	
	\item[] \textbf{Solution} Denote the probability of $x$ such that $r\leq\|x\|_2\leq r^*$ is $\epsilon$, then:
	\begin{equation*}
	P(L_\mathcal{D}(h_r(S))\geq\epsilon) \leq (1-\epsilon)^m \leq e^(-m\epsilon)
	\end{equation*}
Bound it by confidence $\delta$ leads to the conclusion.
	
\item[Ex5] (UML Ex3.4) In this question, we study the hypothesis class of Boolean conjunctions defined as follows. The instance space is $\mathcal{X} = \{0, 1\}^d$ and the label set is $\mathcal{Y} = \{0, 1\}$. A literal over the variables $x_1, \cdots, x_d$ is a simple Boolean function that takes the form $f(\mathbf{x}) = x_i$, for some $i\in [d]$, or $f(\mathbf{x}) = 1-x_i$ for some $i\in [d]$. We use the notation $\overline{x}_i$ as a shorthand for $1-x_i$. A conjunction is any product of literals. In Boolean logic, the product is denoted using the $\wedge$ sign. For example, the function $h(\mathbf{x}) = x_1 \cdot (1 - x_2)$ is written as $x_1 \wedge \overline{x}_2$.

	We consider the hypothesis class of all conjunctions of literals over the $d$ variables. The empty conjunction is interpreted as the all-positive hypothesis (namely, the function that returns $h(\mathbf{x}) = 1$ for all $\mathbf{x}$). The conjunction $x_1 \wedge \tilde{x}_1$ (and similarly any conjunction involving a literal and its negation) is allowed and interpreted as the all-negative hypothesis (namely, the conjunction that returns $h(\mathbf{x}) = 0$ for all $\mathbf{x}$). We assume realizability: Namely, we assume that there exists a Boolean conjunction that generates the labels. Thus, each example $(x, y)\in\mathcal{X}\times\mathcal{Y}$ consists of an assignment to the $d$ Boolean variables $x_1, \cdots, x_d$, and its truth value (0 for false and 1 for true). For instance, let $d = 3$ and suppose that the true conjunction is $x_1 \wedge \overline{x}_2$. Then, the training set $S$ might contain the following instances:
	\begin{equation*}
	((1, 1, 1), 0), ((1, 0, 1), 1), ((0, 1, 0), 0), ((1, 0, 0), 1)
	\end{equation*}

	Prove that the hypothesis class of all conjunctions over $d$ variables is PAC learnable and bound its sample complexity. Propose an algorithm that implements the ERM rule, whose runtime is polynomial in $d\cdot m$.

\item[] \textbf{Solution}: $\mathcal{H}$ is finite, and hence is PAC learnable. Besides the all-negative conjunction, each hypothesis is determined by deciding for each variable $x_i$ with 3 possible choices ( $x_i$,  $\overline{x}_i$ or none). Thus,  $\mathcal{H}=3^d+1$, and 

	\begin{equation*}
	m_\mathcal{H}(\epsilon, \delta) \leq \left\lceil \frac{d\log 3 + \log(1/\delta)}{\epsilon} \right\rceil
	\end{equation*}
	
	Below is the learning algorithm. Start from $h=x_1\wedge\overline{x}_1\wedge\cdots\wedge x_d\wedge\overline{x}_d$, which is always negative. The algorithm does nothing when feed a negative example, and remove $x_1$ or $\overline{x}_1$ when feed a positive example. With realizability assumption, it can labels all training examples correctly and hence is an ERM. Since the algorithm takes linear time (in terms of the dimension $d$) to process each example, the running time is bounded by $O(m\times d)$.
	
\item[Ex6] (UML Ex3.7) The Bayes optimal predictor: Show that for every probability distribution $\mathcal{D}$, the Bayes optimal predictor $f_\mathcal{D}$ is optimal, in the sense that for every classifier $g$ from $X$ to $\{0, 1\}$, $L_\mathcal{D}(f_{\mathcal{D}}) \leq L_\mathcal{D}(g)$.

\item[] \textbf{Solution}: The Bayes predictor labels a sample $x$ according to

	\begin{equation*}
	f_\mathcal{D}(x) = \left\{
	\begin{matrix} 0,& \mathrm{if}\ \mathcal{D}((x,0))\geq \mathcal{D}((x,1)) \\ 1,& \mathrm{otherwise} 
	\end{matrix}\right.
	\end{equation*}
	
	When it labels a sample to be class 0, it holds that $\mathcal{D}((x,0))\geq \mathcal{D}((x,1))$. If the true label function also makes such a decision, then Bayes predictor makes no error. Otherwise, $f(x)=1$, but its probability is no more than 1/2. Any other classifier that labels $x$ to be class 1 will suffer a risk no less than 1/2. Hence, in total, Bayes predictor is the optimal.

\item[Ex7] (UML Ex3.9) Consider a variant of the PAC model in which there are two example oracles: one that generates positive examples and one that generates negative examples, both according to the underlying distribution $\mathcal{D}$ on $\mathcal{X}$. Formally, given a target function $f: \mathcal{X}\rightarrow\{0,1\}$, let $\mathcal{D}^+$ be the distribution over $\mathcal{X}^+ = \{x\in\mathcal{X}: f(x)= 1\}$ defined by $\mathcal{D}^+(A) = \mathcal{D}(A)/\mathcal{D}(\mathcal{X}^+)$, for every $A\in\mathcal{X}^+$. Similarly, $\mathcal{D}^-$ is the distribution over $\mathcal{X}^-$ induced by $\mathcal{D}$. 
	
	The definition of PAC learnability in the two-oracle model is the same as the standard definition of PAC learnability except that here the learner has access to $m_\mathcal{H}^+(\epsilon, \delta)$  i.i.d. examples from $\mathcal{D}^+$ and $m^-(\epsilon, \delta)$ i.i.d. examples from $\mathcal{D}^-$. The learner's goal is to output $h$ s.t. with probability at least $1-\delta$ (over the choice of the two training sets, and possibly over the nondeterministic decisions made by the learning algorithm), both $L(\mathcal{D}^+, f)(h)\leq\epsilon$ and $L(\mathcal{D}^-, f)(h)\leq\epsilon$.
	\begin{itemize}
	\item[7.1] Show that if $\mathcal{H}$ is PAC learnable in the standard one-oracle model, then $H$ is PAC learnable in the two-oracle model.
	\item[7.2] Define $h^+$ to be the always-plus hypothesis and $h^-$ to be the always minus hypothesis. Assume that $h^+, h^-\in \mathcal{H}$. Show that if $\mathcal{H}$ is PAC learnable in the two-oracle model, then $\mathcal{H}$ is PAC learnable in the standard one-oracle model.
	\end{itemize}
	
	\item[] \textbf{Solution}:
	\begin{itemize}
	\item[7.1] Drawing points from the negative and positive oracles with equal provability is equivalent to obtaining i.i.d. examples from a distribution $\mathcal{D}'$ which gives equal probability to positive and negative examples. If we let an algorithm to access to a training set which is drawn i.i.d. according to the $\mathcal{D}'$ with size $m_\mathcal{H}(\epsilon/2, \delta)$, then with probability at least $1-\delta$, it returns $h$ with:
	\begin{equation*}
	\begin{split}
	& \epsilon/2 \geq L_{(\mathcal{D}',f)}(h) = \mathop{\mathbb{P}}\limits_{x\sim\mathcal{D}'}[h(x)\neq f(x)] \\
	& = \mathop{\mathbb{P}}\limits_{x\sim\mathcal{D}'}[f(x)=1, h(x)=0] + \mathop{\mathbb{P}}\limits_{x\sim\mathcal{D}'}[f(x)=0, h(x)=1] \\
	& = \mathop{\mathbb{P}}\limits_{x\sim\mathcal{D}'}[f(x)=1] \cdot \mathop{\mathbb{P}}\limits_{x\sim\mathcal{D}^+}[h(x)=0] + \mathop{\mathbb{P}}\limits_{x\sim\mathcal{D}'}[f(x)=0] \cdot \mathop{\mathbb{P}}\limits_{x\sim\mathcal{D}^-}[h(x)=1] \\
	& = \frac{1}{2}L_{(\mathcal{D}^+,f)}(h) + \frac{1}{2}L_{(\mathcal{D}^-,f)}(h)
	\end{split} 
	\end{equation*}
which implies $L_{(\mathcal{D}^+,f)}(h)\leq\epsilon$ and $L_{(\mathcal{D}^-,f)}(h)\leq\epsilon$.
	\item[7.2] 
	\end{itemize}
	
\item[Ex8] (UML Ex5.3) Prove that if $|\mathcal{X}|\geq km$ for a positive integer $k\geq 2$, then we can replace the lower bound in the No-Free-Lunch theorem. Namely, for the task of binary classification, there exists a distribution $\mathcal{D}\sim\mathcal{X}\times\{0,1\}$ such that:

	\begin{itemize}
	\item[8.1] There exists a function $f:\mathcal{X}\rightarrow\{0,1\}$ with $L_\mathcal{D}(f)=0$.
	\item[8.2] $\mathbb{E}_{S\sim\mathcal{D}^m}[L_\mathcal{D}(A(S))]\geq\frac{1}{2}-\frac{1}{2k}$.
	\end{itemize}

\item[] \textbf{Solution}:
	Only the second proposition should be proved. Similar with the proof in above, 
	\begin{equation*}
	L_{\mathcal{D}_i}(h)=\frac{1}{km}\sum_{x\in C}\mathbbm{1}_{[h(x)\neq f_i(x)]}\geq\frac{1}{km}\sum_{r=1}^p\mathbbm{1}_{[h(v_r)\neq f_i(v_r)]}\geq\frac{k-1}{pk}\sum_{r=1}^p\mathbbm{1}_{[h(v_r)\neq f_i(v_r)]}
	\end{equation*}
 
And similarity,

	\begin{equation*}
	\frac{1}{T}\sum_{i=1}^T L_{\mathcal{D}_i} (A(S_j^i))\geq\frac{k-1}{k}\min_{r\in\{1,\cdots,p\}}\frac{1}{T}\mathbbm{1}_{[A(S^i_j)(v_r)\neq f_i(v_r)]}
	\end{equation*}
	
	So the final bound is $1/2-1/2k$.
\end{itemize} 

\end{itemize}

\end{document}