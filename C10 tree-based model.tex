\documentclass{article}
\usepackage[UTF8]{ctex}
\setmainfont{Calibri Light}
\usepackage{setspace}
\renewcommand{\baselinestretch}{1.2}
\usepackage{amsmath,bm}
\usepackage{framed} 
\usepackage{wrapfig}
\usepackage{amssymb}
\usepackage{ntheorem}
\usepackage{graphicx}
\usepackage{bbm}
\usepackage{hyperref}
\hypersetup{
	colorlinks=true,
	linkcolor=blue,
	filecolor=cyan,      
	urlcolor=red,
	citecolor=green,
}
\newtheorem{theorem}{Theorem}
\newtheorem{corollary}{Corollary}
\newtheorem{lemma}{Lemma}
\newtheorem*{proof}{Proof}
\setlength{\parindent}{2em}
\author{Siheng Zhang\\zhangsiheng@cvte.com}
\title{Chapter \textbf{\textit{6}} Optimization}
\date{\today}
\usepackage[a4paper,left=18mm,right=18mm,top=25mm,bottom=25mm]{geometry} 

\begin{document}
\maketitle  

This part corresponds to \textbf{Chapter 1,3,4 of PRML, Chapter of UML}. It mainly introduces some important properties regarding with functions: convexity, smoothness, strong convexity and Lipschitz, which are basis for the next chapters.

\tableofcontents
\newpage

\section{from decision stump to decision tree}
\section{regression tree}

\newpage
\subsubsection*{P1}
\begin{itemize}
\item[a)] Note that the splitting of bucket $M$ in $T_{old}$ results into the two new buckets $M$ and $M+1$ in $T_{new}$, hence, $R_M=\tilde{R}_M \bigcup \tilde{R}_{M+1}$, i.e., $N_M = \tilde{N}_M + \tilde{N}_{M+1} $, and the former $M-1$ buckets remain the same,

$$C_{imp}(T_{new}) = \sum_{m=1}^{M-1} N_m Q_m(T_{old}) + \tilde{N}_M \tilde{Q}_M(T_{new}) + \tilde{N}_{M+1} \tilde{Q}_{M+1}(T_{new})$$

So, 
\begin{equation*}
\begin{split}
\Delta &= C_{imp}(T_{old}) - C_{imp}(T_{new}) = N_M Q_M(T_{old}) -  \tilde{N}_M \tilde{Q}_M(T_{new}) - \tilde{N}_{M+1} \tilde{Q}_{M+1}(T_{new}) \\
&= \sum_{i:x_i\in R_M} \left(y_i - \frac{1}{N_m} \sum_{i:x_i\in R_M} y_i \right)^2 - \sum_{i:x_i\in \tilde{R}_M} \left(y_i - \frac{1}{\tilde{N}_M} \sum_{i:x_i\in \tilde{R}_M} y_i \right)^2  - \sum_{i:x_i\in \tilde{R}_{M+1}} \left(y_i - \frac{1}{\tilde{N}_{M+1}} \sum_{i:x_i\in \tilde{R}_{M+1}} y_i \right)^2
\end{split}
\end{equation*}
\item[b)] from a), we know that $R_M=\tilde{R}_M \bigcup \tilde{R}_{M+1}$, so

$$\sum_{i:x_i\in \tilde{R}_M} \left(y_i - \frac{1}{\tilde{N}_{M+1}} \sum_{i:x_i\in \tilde{R}_M}  y_i \right)^2 \leq \sum_{i:x_i\in \tilde{R}_M} \left(y_i - \frac{1}{N_m} \sum_{i:x_i\in R_M} y_i\right)^2$$

$$\sum_{i:x_i\in \tilde{R}_{M+1}} \left(y_i - \frac{1}{\tilde{N}_{M+1}} \sum_{i:x_i\in \tilde{R}_{M+1}}  y_i \right)^2 \leq \sum_{i:x_i\in \tilde{R}_{M+1}} \left(y_i - \frac{1}{N_m} \sum_{i:x_i\in R_M} y_i\right)^2$$


And the right-side of two inequalities sum up to $\sum_{i:x_i\in R_M} \left(y_i - \frac{1}{N_m} \sum_{i:x_i\in R_M} y_i \right)^2 $, which leads to $\Delta \geq 0$.

\item[c)] Note that $|T_{old}|=M, |T_{new}|=M+1$. Since $R^2=1-\frac{C_{imp}(T)}{\text{SST}}$, then
\begin{equation*}
\begin{split}
& C_\alpha(T_{new}) \leq C_\alpha(T_{old}) \\
&\iff  C_{imp}(T_{new}) + \alpha M\text{SST} \leq C_{imp}(T_{old}) + \alpha (M+1)\text{SST} \\
&\iff  \frac{C_{imp}(T_{new})}{\text{SST}}  \leq \frac{C_\alpha(T_{old})}{\text{SST}} + \alpha \\
&\iff R^2_{new} - R^2_{old} \geq \alpha
\end{split}
\end{equation*}

\end{itemize}

\end{document}