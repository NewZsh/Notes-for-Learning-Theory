\documentclass{article}
\usepackage{amsmath}
\usepackage{amssymb}
\usepackage{ntheorem}
\usepackage{graphicx}
\usepackage{bbm}
\usepackage{hyperref}
\hypersetup{
	colorlinks=true,
	linkcolor=blue,
	filecolor=cyan,      
	urlcolor=red,
	citecolor=green,
}
\newtheorem{theorem}{Theorem}
\newtheorem{corollary}{Corollary}
\newtheorem{lemma}{Lemma}
\newtheorem*{proof}{Proof}
\setlength{\parindent}{2em}
\author{Siheng Zhang\\zhangsiheng@cvte.com}
\title{Chapter \textbf{\textit{2}} VC-dimension}
\date{\today}      
\usepackage[a4paper,left=18mm,right=18mm,top=25mm,bottom=25mm]{geometry} 

\begin{document}
\maketitle  

This part corresponds to \textbf{Chapter 2-5 in UML}, and mainly answers the following questions:

\begin{itemize}
\item The necessary and sufficient condition of PAC learnability.
\item 
\end{itemize}

\tableofcontents
\newpage

\section{The VC-dimension}

\subsection{Shattering}

Consider the set of threshold functions over the real line $\mathcal{H}=\{h_a(x)=\mathbbm{1}_{[x\leq a]},a\in\mathbb{R}\}$. Let $a^*$ be the threshold such that $L_\mathcal{D}(h^*)=0$. Let $a_0<a^*<a_1$ such that:

\begin{equation*}
\mathop{\mathbb{P}}\limits_{x\sim\mathcal{D}_x}[x\in(a_0,a^*)]=\mathop{\mathbb{P}}\limits_{x\sim\mathcal{D}_x}[x\in(a^*,a_1)]=\epsilon
\end{equation*}

If $\mathcal{D}_x(-\infty,a^*)\leq\epsilon$, we set $a_0=-\infty$, and similarly for $a_1$.

Given a training set $S$, let $b_0=\max\{x:(x,1)\in S\}$ (if no example is positive then $b_0=-\infty>$, and $b_1=\min\{x:(x,0)\in S\}$ (if no example is negative then $b_1=\infty$). Let $b_S$ be the threshold of an ERM hypothesis $h_S$, which implies $b_S\in(b_0,b_1)$, then we have

\begin{equation*}
\mathop{\mathbb{P}}\limits_{S\sim\mathcal{D}^m}[L_\mathcal{D}(h_S)<\epsilon]\leq\mathop{\mathbb{P}}\limits_{S\sim\mathcal{D}^m}[b_0<a_0]+\mathop{\mathbb{P}}\limits_{S\sim\mathcal{D}^m}[b_1>a_1]
\end{equation*}

Each term on the right-side is bounded by $(1-\epsilon)^m\leq e^{-\epsilon m}$. Let $m>\log(2/\delta)/\epsilon$, then the left-side is bounded by $\delta$. As a result, the hypothesis class is PAC-learnable.

The example above shows that: \textbf{finiteness is not a necessary condition for learnability}, and hence we turn to the definition of \textbf{shattering}, which describes the ability of a hypothesis set to cover the training set.

The definition of VC-dimension is motivated from the No-Free-Lunch theorem: without restricting the hypothesis class, for any learning algorithm, an \textbf{adversary} can construct a distribution for which the learning algorithm will perform poorly, while there is another learning algorithm that will succeed on the same distribution. To make any algorithm fail, the \textbf{adversary} used the power of choosing a target function from the set of all possible labelling functions.

When considering PAC learnability of a hypothesis class $\mathcal{H}$, the \textbf{adversary} is restricted to constructing distributions for which some hypothesis $h\in\mathcal{H}$ achieves a zero risk. Since we are considering distributions that are concentrated on elements of $C$, we should study how $h\in\mathcal{H}$ behaves on $C$.

\noindent\textbf{Definition} (Restriction of $\mathcal{H}$ to $C$): The restriction of $\mathcal{H}$ to $C$ is the set of functions from $C$ to $\{0,1\}$ that can be derived from $\mathcal{H}$. That is,
	\begin{equation}
	\mathcal{H}_C=\{(h(c_1),\cdots,h(c_m)):h\in\mathcal{H}\}
	\end{equation}
where we represent each function from $C$ to $\{0,1\}$ as a vector in $\{0,1\}^{|C|}$.

\noindent\textbf{Definition} (Shattering): A hypothesis class $\mathcal{H}$ shatters a finite set $C\in\mathcal{X}$ if the restriction of $\mathcal{H}$ to $C$ is the set of all functions from $C$ to $\{0,1\}$. That is, $|\mathcal{H}_C|=2^{|C|}$.

\begin{corollary*} Let $\mathcal{H}$ be a hypothesis class of functions from $\mathcal{X}$ to $\{0, 1\}$. Let $m$ be a training set size. Assume that there exists a set $C\subset\mathcal{X}$ of size $2m$ that is shattered by $\mathcal{H}$. Then, for any learning algorithm, $A$, there exist a distribution $\mathcal{D}$ over $\mathcal{X} \times \{0, 1\}$ and a predictor $h\in\mathcal{H}$ such that $L_{\mathcal{D}}(h) = 0$ but with probability of at least 1/7 over the choice of $S\sim\mathcal{D}^m$ we have that $L_\mathcal{D}(A(S))\geq 1/8$.
\end{corollary*}

The corollary shows that \textbf{whenever if $\mathcal{H}$ shatters some set $\mathcal{C}$ of size $2m$, then we cannot learn $\mathcal{H}$ by using $m$ examples.} This leads us directly to the definition of the VC dimension.

\subsection{The VC-dimension}
\textbf{Definition} (VC-dimension): The VC-dimension of a hypothesis class $\mathcal{H}$, denoted $\mathrm{VCdim}(\mathcal{H})$, is the maximal size of a set $C\subset\mathcal{X}$ that can be shattered by $\mathcal{H}$. If $\mathcal{H}$ can shatter sets of arbitrarily large size we say that $\mathcal{H}$ has infinite VC-dimension.

\begin{theorem}
If $\mathcal{H}$ is a class of infinite VC-dimension, then $\mathcal{H}$ is not PAC learnable.
\end{theorem}

\subsubsection{Examples}

	To calculate the VC-dimension for a hypothesis set, we should show that:
	\begin{itemize}
	\item There \textbf{exists} a subset of size $d$ that can be shattered;
	\item \textbf{Every} subset of size $d+1$ can not be shattered.
	\end{itemize}
	
	\begin{itemize}
	\item [\textbf{1}] Threshold functions
		\begin{equation*}
		\mathcal{H}=\{\mathbbm{1}_{x\leq a}:a\in\mathbb{R}\}
		\end{equation*}
		
	For an arbitrary set $C=\{c\}$, $\mathcal{H}$ shatters $C$, therefore $\mathrm{VCdim}(\mathcal{H})\geq 1$; for an arbitrary set $C=\{c_1,c_2\}$, where $c_1\leq c_2$, any threshold that assigns 0 to $c_1$ must assign 0 to $c_2$. In other words, not all functions from $\mathcal{C}$ to $\{0,1\}$ are included by $\mathcal{H}_C$. So, $\mathcal{H}$ does not shatter $C$.
	
	\item [\textbf{2}] Intervals
		\begin{equation*}
		\mathcal{H}=\{\mathbbm{1}_{x\in(a,b)}:a<b,a,b\in\mathbb{R}\}
		\end{equation*}
	
	Denote the set $C=\{c_1, c_2\}$. If we take $a>c_2$ or $b<c_2$, the we have $h_{a,b}(c_1)=0, h_{a,b}(c_2)=0$; if we take $c_1<a<c_2<b$, the we have $h_{a,b}(c_1)=0, h_{a,b}(c_2)=1$; if we take $a<c_1<b<c_2$, the we have $h_{a,b}(c_1)=1, h_{a,b}(c_2)=0$; if we take $a<c1<c2<b$, then we have $h_{a,b}(c_1)=1, h_{a,b}(c_2)=1$. Therefore, $\mathcal{H}_C$ is the set of all functions from $C$ to $\{0,1\}^2$.
		
	Take the set $C=\{c_1,c_2,c_3\}$, without loss of generalization, let the labels be  $(1,0,1)$, therefore $\mathcal{H}$ does no shatter $C$.
	
	Hence, $\mathrm{VCdim}(\mathcal{H})=2$.
	
	\item [\textbf{3}] Axis Aligned Rectangles
		\begin{equation*}
		\mathcal{H}=\{\mathbbm{1}_{(a_1\leq x_1\leq a_2,b_1\leq x_2\leq b_2)}:a_1<a_2,b_1<b_2\}
		\end{equation*}
	
	Any set with 4 points can be shattered. Take the set with 5 points. Suppose that there is 1 point (labelled as 0) surrounded by 4 points (labelled as 1), it cannot be shattered. Hence, $\mathrm{VCdim}(\mathcal{H})=4$.
	
	\item [\textbf{4}] Finite class
	
	Let $\mathcal{H}$ be a finite class. Then, clearly, for any set $C$ we have $|\mathcal{H}_C|\leq|\mathcal{H}|$ and thus it cannot be shattered if $|\mathcal{H}|<2^{|C|}$. This implies that $\mathrm{VCdim}(\mathcal{H})<\log_2|\mathcal{H}|$.
	
	\textit{\underline{remark1}}: In the previous examples, the VC-dimension happened to equal the number of parameters defining.  This is not always true. See exercise ? for detail.
	\end{itemize}

\section{Fundamental theorem of PAC learning}

	\begin{theorem}
	Let $\mathcal{H}$ be a hypothesis class of functions from a domain $\mathcal{X}$ to $\{0,1\}$ and let the loss function be the 0-1 loss. Then, the following are equivalent:
	
	1. The hypothesis class has uniform convergence property.

	\begin{equation}
	m_\mathcal{H}^{UC}(\epsilon,\delta)=O\left(\frac{d+\log(1/\delta)}{\epsilon^2}\right)	
	\end{equation}

	2. Any ERM rule is a successful agnostic PAC learner for the hypothesis class.
	
	3. The hypothesis class is agnostic PAC learnable.
	
	\begin{equation}
	m_\mathcal{H}(\epsilon,\delta)=O\left(\frac{d+\log(1/\delta)}{\epsilon^2}\right)	
	\end{equation}

	4. The hypothesis class is PAC learnable.

	\begin{equation}
	m_\mathcal{H}(\epsilon,\delta)=O\left(\frac{d\log(1/\epsilon)+\log(1/\delta)}{\epsilon}\right)
	\end{equation}

	5. Any ERM rule is a successful PAC learner for the hypothesis class.
	
	6. The hypothesis class has a finite VC-dimension.
	\end{theorem}
	
	\textit{remark1: 1->2->3->4->5->6 are all learned. The leaving part is 6->1, which is solved below.}

\section{Effective size of a hypothesis class}

\section{Non-uniform learnability}

“non-uniform learnability” allows the sample size to be non-uniform with respect to the different hypotheses with which the learner is competing. 

A hypothesis is $(\epsilon, \delta)$-competitive with another if
\section{Summary}
%
%\begin{enumerate}
%\item ...
%\end{enumerate}


\section{Exercises and solutions}

\textit{
      To be continue...\\
      Chapter 3. Bayesian-PAC\\
      Chapter 4. Generalization in Deep Learning}

\end{document}