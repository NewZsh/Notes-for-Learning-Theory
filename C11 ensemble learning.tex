\documentclass{article}
\usepackage[UTF8]{ctex}
\setmainfont{Calibri Light}
\usepackage{setspace}
\renewcommand{\baselinestretch}{1.2}
\usepackage{amsmath,bm}
\usepackage{framed} 
\usepackage{wrapfig}
\usepackage{amssymb}
\usepackage{ntheorem}
\usepackage{graphicx}
\usepackage{bbm}
\usepackage{hyperref}
\hypersetup{
	colorlinks=true,
	linkcolor=blue,
	filecolor=cyan,      
	urlcolor=red,
	citecolor=green,
}
\newtheorem{theorem}{Theorem}
\newtheorem{corollary}{Corollary}
\newtheorem{lemma}{Lemma}
\newtheorem*{proof}{Proof}
\setlength{\parindent}{2em}
\author{Siheng Zhang\\zhangsiheng@cvte.com}
\title{Chapter \textbf{\textit{6}} Optimization}
\date{\today}
\usepackage[a4paper,left=18mm,right=18mm,top=25mm,bottom=25mm]{geometry} 

\begin{document}
\maketitle  

This part corresponds to \textbf{Chapter 1,3,4 of PRML, Chapter of UML}. It mainly introduces some important properties regarding with functions: convexity, smoothness, strong convexity and Lipschitz, which are basis for the next chapters.

\tableofcontents
\newpage

\section{Gradient descent}

	\begin{framed}
	\begin{scriptsize}
	\begin{spacing}{1.2}
	\noindent\textit{\textbf{remark1.} 1}
	\end{spacing}
	\end{scriptsize}
	\end{framed}

\section{Sub-gradient descent}
\section{Stochastic gradient descent}
\section{Adaptive algorithms}
\end{document}