\documentclass{article}
\usepackage{amsmath}
\usepackage{amssymb}
\usepackage{ntheorem}
\usepackage{graphicx}
\usepackage{bbm}
\newtheorem{theorem}{Theorem}
\newtheorem{corollary}{Corollary}
\newtheorem{lemma}{Lemma}
\newtheorem*{proof}{Proof}

\author{Siheng Zhang\\zhangsiheng@cvte.com}
\title{Chapter THREE Generative Models}
\date{\today}      
\usepackage[a4paper,left=18mm,right=18mm,top=25mm,bottom=25mm]{geometry} 

\begin{document}
\maketitle  

The notes is mainly based on the following books:

\begin{itemize}
\item Understanding Machine Learning: From Theory to Algorithms, Shai Shalev-Shwartz and Shai Ben-David, 2014 \footnote{https://www.cs.huji.ac.il/\~shais/UnderstandingMachineLearning/understanding-machine-learning-theory-algorithms.pdf}

\item Pattern Recognition and Machine Learning, Christopher M. Bishop, 2006 \footnote{http://users.isr.ist.utl.pt/\~wurmd/Livros/school/Bishop\ -\ Pattern\ Recognition\ And\ Machine\ Learning\ -\ Springer\ \ 2006.pdf}

\item Probabilistic Graphical Models: Principles and Techniques, Daphne Koller and Nir Friedman, 2009 \footnote{https://mitpress.mit.edu/books/probabilistic-graphical-models}

\item Graphical Models, Exponential Families, and Variational Inference, Martin J. Wainwright and Michael I. Jordan, 2008 \footnote{https://people.eecs.berkeley.edu/\~wainwrig/Papers/WaiJor08\_FTML.pdf}
\end{itemize}

This part corresponds to \textbf{Chapter 24, 31 in UML, Chapter ? in PRML, Chapter ? in PGM}, and mainly answers the following questions:

\begin{itemize}
\item 
\item 
\end{itemize}

\newpage

\tableofcontents
\newpage

\section{Naive Bayes}

\section{Density estimation}
	\subsection{Parametric methods}
	\subsection{Non-parametric methods}

\section{Bayesian Reasoning}

\section{PAC-Bayes}

\section{Generative models}
	\subsection{GMM (Gaussian mixture models)}
	\subsection{HMM (Hidden Markov models)}
	\subsection{\textit{v.s.} discriminant models}
	\subsection{Naive Bayes to linear discriminant models}


\section{Exercises and solutions}

\textit{
      Chapter 4. Linear models, perceptron, MLP, deep learning, Generalization bounds on deep learning.}

\end{document}